\documentclass[a4paper, 12pt]{article}

% Language setting
\usepackage[italian]{babel}

% Set page size and margins
\usepackage[a4paper,top=2cm,bottom=2cm,left=3cm,right=3cm,marginparwidth=1.75cm]{geometry}

% Useful packages
\usepackage[utf8]{inputenc}
\usepackage[T1]{fontenc}
\usepackage[italian]{babel}
\usepackage{graphicx}
\usepackage{listings}
\usepackage{xcolor}
\usepackage{hyperref}
\usepackage{caption}
\usepackage{subcaption}

\hypersetup{
pdftitle={Progetto Tetris Arduino},
pdfauthor={Enrico Ferraiolo}
}

\title{\textbf{RELAZIONE: \\ Tetris Arduino}}
\author{Enrico Ferraiolo 0001191698}
\date{}

\begin{document}

\maketitle

\begin{center}
    \textbf{Laurea Magistrale in Informatica}\\
    \vspace{0.3cm}
    Corso: Laboratorio di Making\\
    a.a. 2024-2025
    \vspace{2cm}
\end{center}

\newpage

\tableofcontents
\newpage

\section{Introduzione}
Il gioco Tetris è uno dei puzzle game più celebri di sempre: l'utente deve ruotare e spostare pezzi geometrici ("tetramini") che cadono, 
completando linee orizzontali per ottenere punti.\\
L'obiettivo di questo progetto è realizzare una versione giocabile su Arduino di Tetris, utilizzando un display a matrice LED 8x8 (MAX7219) 
per il campo di gioco, un display LCD 16x2 per visualizzare punteggio e stato, mentre i controlli sono gestiti tramite telecomando IR e encoder rotativo.\\
\newpage

\section{Componenti Hardware}
\subsection{Microcontrollore}
Il microcontrollore utilizzato è il \textbf{Elegoo UNO R3}, esso è un'alternativa compatibile all'Arduino UNO.

\subsection{Display a matrice LED - Campo di Gioco}
Il display a matrice LED è un modulo \textbf{MAX7219}, nel caso specifico del progetto in questione è stato utilizzato un modulo 8x8.
Ogni LED della matrice può essere acceso o spento in modo indipendente, permettendo di visualizzare il campo di gioco e i tetramini.\\
Ogni LED rappresenta una cella del campo di gioco.

\subsection{Display LCD - Informazioni di Gioco}
Il display LCD è un modulo 16x2, esso è utilizzato per visualizzare il punteggio e lo stato del gioco.\\
È stato utilizzato il modulo \textbf{LCD 1602}.
A schermo vengono visualizzati:
\begin{itemize}
    \item \textbf{Punteggio}: il punteggio attuale del giocatore
    \item \textbf{Stato}: lo stato del gioco (in corso, in pausa, finito)
    \item \textbf{Velocità}: la velocità attuale del gioco
    \item \textbf{Istruzioni ausiliarie}: istruzioni per il giocatore
\end{itemize}

\subsection{Controlli}
Il progetto prevede l'utilizzo di un telecomando IR e di un encoder rotativo per il controllo del gioco.\\

\subsubsection{Controlli Infrarossi}
Il telecomando IR è un dispositivo che consente di inviare segnali preimpostati a distanza tramite infrarossi.\\
Sono stati utilizzati un telecomando IR e un ricevitore IR compatibili.\\
Il telecomando IR è dotato di diversi tasti, ognuno dei quali invia un codice univoco quando premuto.\\
I tasti utilizzati nel progetto sono:
\begin{itemize}
    \item \textbf{POWER}: per accendere e spegnere il gioco
    \item \textbf{FAST BACK}: per muovere il tetramino a sinistra
    \item \textbf{FAST FORWARD}: per muovere il tetramino a destra
    \item \textbf{PAUSE}: per mettere in pausa il gioco
    \item \textbf{VOL+}: per aumentare la velocità del gioco
    \item \textbf{VOL-}: per diminuire la velocità del gioco
\end{itemize}
\subsubsection{Encoder Rotativo}
L'encoder rotativo è utilizzato per il controllo della direzione e della velocità del gioco.\\
\begin{itemize}
    \item \textbf{Rotazione in senso orario}: aumenta la velocità del gioco
    \item \textbf{Rotazione in senso antiorario}: diminuisce la velocità del gioco
\end{itemize}



\end{document}