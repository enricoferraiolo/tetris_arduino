\documentclass[a4paper, 12pt]{article}

% Language setting
\usepackage[italian]{babel}

% Set page size and margins
\usepackage[a4paper,top=2cm,bottom=2cm,left=3cm,right=3cm,marginparwidth=1.75cm]{geometry}

% Useful packages
\usepackage[utf8]{inputenc}
\usepackage[T1]{fontenc}
\usepackage[italian]{babel}
\usepackage{graphicx}
\usepackage{listings}
\usepackage{xcolor}
\usepackage{hyperref}
\usepackage{caption}
\usepackage{subcaption}
\usepackage{booktabs}
\usepackage{makecell}
\usepackage{amsmath}
\usepackage{amssymb}
\usepackage{color}
\usepackage{float}
\newcommand{\RedBlock}{\colorbox{red}{\rule{1.2ex}{1.2ex}}}      % blocco pieno rosso
\newcommand{\WhiteBlock}{\fcolorbox{black}{white}{\rule{1.2ex}{1.2ex}}} % blocco vuoto
\hypersetup{
pdftitle={Progetto Tetris Arduino},
pdfauthor={Enrico Ferraiolo}
}

\title{\textbf{RELAZIONE: \\ Tetris Arduino}}
\author{Enrico Ferraiolo 0001191698}
\date{}

\begin{document}

\maketitle

\begin{center}
    \textbf{Laurea Magistrale in Informatica}\\
    \vspace{0.3cm}
    Corso: Laboratorio di Making\\
    a.a. 2024-2025
    \vspace{2cm}
\end{center}

\newpage

\tableofcontents
\newpage

\section{Introduzione}
Il gioco Tetris è uno dei puzzle game più celebri di sempre: l'utente deve ruotare e spostare pezzi geometrici ("tetramini") che cadono,
completando linee orizzontali per ottenere punti.\\
L'obiettivo di questo progetto è realizzare una versione giocabile su Arduino di Tetris, utilizzando un display a matrice LED 8x8 (MAX7219)
per il campo di gioco, un display LCD 16x2 per visualizzare punteggio e stato, mentre i controlli sono gestiti tramite telecomando IR e encoder rotativo.\\
Lo scopo del progetto è quello di realizzare una versione del gioco Tetris su Arduino totalmente funzionante con diversi moduli di input e output.\\
\newpage

\section{Componenti Hardware}
Di seguito vengono elencati e descritti i componenti hardware utilizzati per il progetto.

\subsection{Microcontrollore}
Il microcontrollore utilizzato è il \textbf{Elegoo UNO R3}, esso è un'alternativa compatibile all'Arduino UNO.

\subsection{Display a matrice LED - Campo di Gioco}
Il display a matrice LED è un modulo \textbf{MAX7219}, nel caso specifico del progetto in questione è stato utilizzato un modulo 8x8.
Ogni LED della matrice può essere acceso o spento in modo indipendente, permettendo di visualizzare il campo di gioco e i tetramini.\\
Ogni LED rappresenta una cella del campo di gioco.

\subsection{Display LCD - Informazioni di Gioco}
Il display LCD è un modulo 16x2, esso è utilizzato per visualizzare il punteggio e lo stato del gioco.\\
È stato utilizzato il modulo \textbf{LCD 1602}.
A schermo vengono visualizzati:
\begin{itemize}
    \item \textbf{Punteggio}: il punteggio attuale del giocatore
    \item \textbf{Stato}: lo stato del gioco (in corso, in pausa, finito)
    \item \textbf{Velocità}: la velocità attuale del gioco
    \item \textbf{Istruzioni ausiliarie}: istruzioni per il giocatore
\end{itemize}

\subsection{Controlli}
Il progetto prevede l'utilizzo di un telecomando IR e di un encoder rotativo per il controllo del gioco.\\

\subsubsection{Controlli Infrarossi}
Il telecomando IR è un dispositivo che consente di inviare segnali preimpostati a distanza tramite infrarossi.\\
Sono stati utilizzati un telecomando IR e un ricevitore IR compatibili.\\
Il telecomando IR è dotato di diversi tasti, ognuno dei quali invia un codice univoco quando premuto.\\
I tasti utilizzati nel progetto sono:
\begin{itemize}
    \item \textbf{POWER}: per accendere e spegnere il gioco
    \item \textbf{FAST BACK}: per muovere il tetramino a sinistra
    \item \textbf{FAST FORWARD}: per muovere il tetramino a destra
    \item \textbf{PAUSE}: per mettere in pausa il gioco
    \item \textbf{VOL+}: per aumentare la velocità del gioco
    \item \textbf{VOL-}: per diminuire la velocità del gioco
\end{itemize}
\subsubsection{Encoder Rotativo}
L'encoder rotativo è utilizzato per il controllo della direzione e della velocità del gioco.\\
\begin{itemize}
    \item \textbf{Rotazione in senso orario}: aumenta la velocità del gioco
    \item \textbf{Rotazione in senso antiorario}: diminuisce la velocità del gioco
\end{itemize}

\section{Il Gioco}
Il gioco Tetris è un puzzle game in cui il giocatore deve ruotare e spostare tetramini che cadono dall'alto, quest'ultimi sono composti da
4 celle e possono essere ruotati e spostati a sinistra o a destra nel campo di gioco.\\
Il giocatore deve completare linee orizzontali per ottenere punti e quando una linea è completata, essa scompare e il punteggio aumenta.\\
Il gioco termina quando i tetramini raggiungono la parte superiore del campo di gioco e non c'è quindi più spazio per far cadere nuovi tetramini.\\

\subsection{Tetramini}
Tra i tetramini presenti nel gioco implementato in questo progetto troviamo le seguenti forme presenti nella tabella \ref{tab:tetramini}:
\begin{table}[H]
    \centering
    \caption{Rappresentazione dei tetramini}
    \label{tab:tetramini}
    \begin{tabular}{
            >{\bfseries}l   % Nome
            c               % Codici binari
            c               % Dimensioni
            l               % Rappresentazione
        }
        \toprule
        Pezzo & Codici binari                                      & W\(\times\)H & Forma \\
        \midrule
        I     & \makecell{\texttt{0b1111}                                                 \\\texttt{0b0000}\\\texttt{0b0000}\\\texttt{0b0000}}
              & 4\(\times\)1
              & \makecell{\RedBlock\RedBlock\RedBlock\RedBlock}                           \\
        \addlinespace
        J     & \makecell{\texttt{0b0111}                                                 \\\texttt{0b0100}}
              & 3\(\times\)2
              & \makecell{\WhiteBlock\RedBlock\RedBlock\RedBlock                          \\\WhiteBlock\RedBlock\WhiteBlock\WhiteBlock} \\
        \addlinespace
        L     & \makecell{\texttt{0b1110}                                                 \\\texttt{0b0010}}
              & 3\(\times\)2
              & \makecell{\RedBlock\RedBlock\RedBlock\WhiteBlock                          \\\WhiteBlock\RedBlock\WhiteBlock\WhiteBlock} \\
        \addlinespace
        O     & \makecell{\texttt{0b0110}                                                 \\\texttt{0b0110}}
              & 2\(\times\)2
              & \makecell{\WhiteBlock\RedBlock\RedBlock\WhiteBlock                        \\\WhiteBlock\RedBlock\RedBlock\WhiteBlock} \\
        \addlinespace
        S     & \makecell{\texttt{0b0111}                                                 \\\texttt{0b0010}}
              & 3\(\times\)2
              & \makecell{\WhiteBlock\RedBlock\RedBlock\RedBlock                          \\\WhiteBlock\WhiteBlock\RedBlock\WhiteBlock} \\
        \addlinespace
        T     & \makecell{\texttt{0b1100}                                                 \\\texttt{0b0110}}
              & 3\(\times\)2
              & \makecell{\RedBlock\RedBlock\WhiteBlock\WhiteBlock                        \\\WhiteBlock\RedBlock\RedBlock\WhiteBlock} \\
        \addlinespace
        Z     & \makecell{\texttt{0b1110}                                                 \\\texttt{0b1000}}
              & 3\(\times\)2
              & \makecell{\RedBlock\RedBlock\RedBlock\WhiteBlock                          \\\RedBlock\WhiteBlock\WhiteBlock\WhiteBlock} \\
        \bottomrule
    \end{tabular}
\end{table}

\section{Ambienti di Sviluppo}
Il progetto è stato sviluppato per essere eseguito su:
\begin{itemize}
    \item \textbf{Hardware fisico}: scheda compatibile e moduli connessi
    \item \textbf{Simulatore}: per testare il codice senza hardware fisico su un simulatore software
\end{itemize}
Per cambiare ambiente di sviluppo è sufficiente cambiare la variabile \texttt{PRODUCTION} in nel file sorgente: \texttt{PRODUCTION = true} per l'hardware fisico e \texttt{PRODUCTION = false} per il simulatore.\\
Questo serve perché i codici infrarossi inviati dal telecomando IR sono diversi a seconda dell'ambiente di sviluppo.\\

\section{Setup Hardware}
Questa sezione riporta e descrive come sono stati connessi i vari moduli hardware al microcontrollore.

\subsection{Display a matrice LED (MAX7219)}
Il modulo MAX7219 a matrice 8x8 viene utilizzato per visualizzare il campo di gioco e i tetramini.\\
Di seguito sono riportate le connessioni tra il modulo e il microcontrollore.
\begin{table}[H]
    \centering
    \caption{Connessioni Matrix (MAX7219) - Microcontrollore}
    \label{tab:matrix-max7219-connections}
    \begin{tabular}{ll}
        \toprule
        \textbf{Matrix Pin} & \textbf{Microcontrollore Pin} \\
        \midrule
        VCC                 & 5V                            \\
        GND                 & GND                           \\
        DIN                 & Pin 12                        \\
        CS                  & Pin 10                        \\
        CLK                 & Pin 11                        \\
        \bottomrule
    \end{tabular}
\end{table}

\subsection{Display LCD 16x2}
Il modulo LCD 1602 serve per mostrare punteggio, stato e velocità di gioco.\\
Di seguito sono riportate le connessioni tra il modulo e il microcontrollore.
\begin{table}[H]
    \centering
    \caption{Connessioni LCD Display (16x2) - Microcontrollore}
    \label{tab:lcd-16x2-connections}
    \begin{tabular}{ll}
        \toprule
        \textbf{LCD Pin} & \textbf{Microcontrollore Pin}                  \\
        \midrule
        RS               & Pin 13                                         \\
        E                & Pin 9                                          \\
        D4               & Pin 6                                          \\
        D5               & Pin 5                                          \\
        D6               & Pin 7                                          \\
        D7               & Pin 4                                          \\
        VSS              & GND                                            \\
        VDD              & 5V                                             \\
        RW               & GND                                            \\
        A (Anodo)        & 5V (attraverso un resistore da 220 \(\Omega\)) \\
        K (Catodo)       & GND                                            \\
        \bottomrule
    \end{tabular}
\end{table}

\subsection{Ricevitore Infrarossi (IR)}
Il modulo ricevitore IR decodifica i codici inviati dal telecomando per gestire i comandi di gioco.\\
Di seguito sono riportate le connessioni tra il modulo e il microcontrollore.
\begin{table}[H]
    \centering
    \caption{Connessioni IR Receiver Module - Microcontrollore}
    \label{tab:ir-receiver-connections}
    \begin{tabular}{ll}
        \toprule
        \textbf{IR Pin} & \textbf{Microcontrollore Pin} \\
        \midrule
        VCC             & 5V                            \\
        GND             & GND                           \\
        OUT/Data        & Pin 3                         \\
        \bottomrule
    \end{tabular}
\end{table}

\subsection{Encoder Rotativo}
L'encoder rotativo serve per regolare manualmente la velocità di caduta dei tetramini.\\
Di seguito sono riportate le connessioni tra il modulo e il microcontrollore.
\begin{table}[H]
    \centering
    \caption{Connessioni Rotary Encoder Module - Microcontrollore}
    \label{tab:rotary-encoder-connections}
    \begin{tabular}{ll}
        \toprule
        \textbf{Encoder Pin} & \textbf{Microcontrollore Pin} \\
        \midrule
        CLK                  & Pin 2                         \\
        DT                   & Pin 8                         \\
        SW (Switch)          & Non usato                     \\
        VCC                  & 5V                            \\
        GND                  & GND                           \\
        \bottomrule
    \end{tabular}
\end{table}




\end{document}